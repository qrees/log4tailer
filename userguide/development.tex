\part{Development}
\section{Hacking}
If you would like to collaborate, just let me know. I'm taking this project seriously now and 
any new feature will require unit testing. This project was originally a multicolored tailer and I just 
did not bother about unit testing as it was short in code and did basically what I wanted. This project is 
going to grow significantly and I just do not want to break any functionality that already works. As of this 
release I've been using unittest and mox (mocking) libraries. 

\logftailer{} began in launchpad, but as I was working on my own, I moved it to googlecode. If more people joins 
in this project, it could be moved to launchpad, as it is more suited for bigger opensource teams.

\section{Class Diagram}
In the next figure we can take a look at what is the relationship between classes in the log4tailer 
project:

\begin{figure}[htp]
\centering
\includegraphics[scale=0.60]{log4tailerClassDiagram.png}
\caption{\logftailer{} class diagram}\label{fig:class diagram}
\end{figure}



\section{Further work\footnote{work that could or could not be implemented in future releases.}}

\logftailer{} began being just a multicolored tailer, but it has been growing in functionalities and new 
concepts. With the introduction of \emph{Actions}, \logftailer{} can be expanded to use more functionalities. 
Features that I've been thinking of:
\begin{enumerate}
 \item web frontend action
 \item ncurses action
 \item nagios action
\item \emph{Analytics} will introduce log behaviour analysis based on patterns and log statistical analysis.
 \item \emph{Log4Servers} and \emph{Log4Clients} will enable having log4tailers being run in remote servers and clients 
will be able to communicate to those servers and trigger a decission based on the log4tailers action outcomes.
\item tailing remotely from your desktop by means of ssh.
\item Inactivity action could notify by using other actions like emailing. As of now it uses printing. Actually, 
inactivity monitoring is not a notification action, so maybe I'll make a package for monitoring stuff to 
separate the concept of notification actions vs monitoring actions.
\item I'll write an optional installer to install \logftailer{} system wide, but you will be able to run it 
in standalone mode.
\item A GTK gui
\end{enumerate}
\newpage
\part{Log4Server}

\section{A centralized reporting web application}
The \logftailer{} project includes a web backend application that receives notifications 
from the log4tailer clients, notifying in a web front page about the status of the logs 
in several machines. The log4server is implemented using the Django web framework and can run 
in any wsgi compliant web server, such as Apache, Cherokee, Nginx or CherryPy, just to name 
a few. Basically, the clients will register first to the server and then notify if any fatal, 
error, critical or target logtrace has been found. 

A network diagram is showed in figure \autoref{fig:serverdiagram}:

\begin{figure}[hb]
\centering
\includegraphics[scale=0.50]{serverdiagram.png}
\caption{Log4Server network diagram}\label{fig:serverdiagram}
\end{figure}


In the next section, we will describe the API that the log4server application implements and the 
one being used by the log4tailer poster notification.

\section{Log4Tailer compliant server}
\logftailer{} client has the poster notification that allows the tailer to communicate 
with a centralized web application, notifying it of possible problems from remote logs. 
The poster notification will first register the tailer client to the server, and next 
calls will be for alert notifications only. The compliant API interface is as follows:

\subsection{Registration}

Log4tailer client registers to the server on startup when poster notification is provided.

\begin{flushleft}
 \begin{tabular}{|c|c|l|}
 \hline 
 \rowcolor{cyan} {\color{white} \textit{\textbf{HTTP Method}}} &  {\color{white} 
  \textit{\textbf{URL}}}  & {\color{white} 
 \textit{\textbf{Description}}}\\
 POST & /register & Log4tailer client registration to the server\\
 \hline
\end{tabular}
\end{flushleft}
The POST method will have in the body a JSON object with the following information:

\begin{itemize}
 \item logpath Full path of the log 
 \item hostname log's server hostname
\end{itemize}

\noindent
Upon a successul POST the server will reply a 201 CREATED answer.

\noindent
Example:

\begin{codeexample}

POST /register

 \{``logpath'' : ``/var/log/messages'', ``logserver'' : ``localhost''\} 

RESPONSE

 \{``id'' : 3\} 

HTTP/1.1 201 CREATED.
\end{codeexample}

The id returned is the log identifier id.

\subsection{Alerting}

\begin{flushleft}
 \begin{tabular}{|c|c|l|}
 \hline 
 \rowcolor{cyan} {\color{white} \textit{\textbf{HTTP Method}}} &  {\color{white} 
  \textit{\textbf{URL}}}  & {\color{white} 
 \textit{\textbf{Description}}}\\
 POST & /alert & New alert has been found.\\
 \hline
\end{tabular}
\end{flushleft}
The POST method will have in the body a JSON object with the following information:

\begin{itemize}
 \item logtrace logtrace that triggered the alert.
 \item level level of the aforementioned logtrace.
 \item log log where the logtrace belongs to.
    \begin{itemize}
      \item logpath Full path of the log 
      \item hostname log's server hostname
    \end{itemize}
\end{itemize}
Upon a successul POST the server will reply a 201 CREATED answer.

\noindent
Example:

\begin{codeexample}

POST /alerts

 \{``logtrace'' : ``This is an error trace'', 
   ``loglevel'' : ``error'',
   ``log'' : \{``id'' : ``logid'', ``logpath'' : ``/var/log/messages'', ``logserver'': ``192.168.1.1''\}\} 

RESPONSE

HTTP/1.1 201 CREATED.
\end{codeexample}

\subsection{Status}

\begin{flushleft}
 \begin{tabular}{|c|c|l|}
 \hline 
 \rowcolor{cyan} {\color{white} \textit{\textbf{HTTP Method}}} &  {\color{white} 
  \textit{\textbf{URL}}}  & {\color{white} 
 \textit{\textbf{Description}}}\\
 GET & /status & Status of the log files.\\
 \hline
\end{tabular}
\end{flushleft}
It returns the last 10 log traces triggered along with the log and server they belong to. The 
date when it happened is reported as well.

\noindent
Example:

\begin{codeexample}

GET /alerts/status

RESPONSE

HTTP/1.1 200 OK.
\end{codeexample}


\noindent
If you go to /alerts/status in your web browser, you would see something as showed in figure 
\autoref{fig:logstatus}:

\begin{figure}[ht]
%\centering
\includegraphics[scale=0.50]{logstatus.png}
\caption{Log4Server status web reporting}\label{fig:logstatus}
\end{figure}

\section{Log4Server deployment}
Log4Server is implemented using the Django web framework. It persists alertable log traces into a 
database for easy reporting and log4tailer client registrations. 

The application uses buildout to manage all application dependencies and deployment. By issuing 
the bin/buildout command, it will build a log4server.wsgi file, that is the one that will need to 
be used when setting up the web server. In this document, we will show the instructions on how to 
set it up by using the Apache web server. 

\subsection{Log4Server deployment dependencies}

You'll need to install a web server such as Apache listening on the port you want Log4Server to listen 
to. Make sure to setup the clients to post notifications to that port. Apart from that, a Django 
compatible Python runtime should be available in the server as well. Python 2.6 runtime 
is the recommended one. 

\subsection{Deciding on database}

Log4Server needs a database in order to persist the alertable logtraces. Sqlite3, Mysql or PostGres 
should be, by far, enough. If you use Sqlite3, make sure that Apache process has write permissions 
on the database file. 

\subsection{Apache setup}

You'll need to setup a VirtualHost for that and make sure that the mod\_wsgi module is available 
for Apache. 

\newpage

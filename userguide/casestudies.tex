\part{Case Studies}
\section{Full Automatic Log Monitorization}
Full Automatic log monitorization can be performed when you execute \logftailer{} 
in silent mode passing the parameters \emph{-s}. 
Log4Tailer will run silently in the background notifying by email 
when something goes wrong. As of now, it will notify errors, fatals and those targets 
specified as a parameter or in the config file. It is important to notice that every log 
can have its own set of targets (regexes). Apart from that, you can make log4tailer 
to monitor inactivity in the log and notify you by email as well. You just need to specify 
that in the config file as explained in the section \ref{sec:inactivitysection}.

Summing up, full automatic monitorization will monitor inactivity, errors, fatals and targets specified 
in the config file or command line as parameters. This will give you extra confidence on the 
monitoring of your application if your application uses already nagios or other monitoring software.
\footnote{It is important to notice that pausemodes should not be enabled. That feature is 
to pause the output when having PrintAction enabled.}

\section{Semi Automatic Log Monitorization}
You can have a mix of email notification and normal colorized print action. You just 
need to execute log4tailer passing as a parameter \emph{-m} and the 
corresponding configfile if you want to enable additional features. 

\section{No SMTP email access for log4tailer}
Sometimes a server can have the email ports closed (firewalled) due to security policies. 
Alternatives:
\begin{itemize}
\item For those cases you can use the executor notification using the mail Linux command line to 
send email provided that the server runs some 
kind of MTA like sendmail. Let's see an example:
\begin{cmd}
 log4tail --executable -c configfile.txt /var/log/out.log
\end{cmd}
where in the configfile.txt you could write something like\footnote{If you use the 
\emph{echo} command line tool providing both place holders (log trace, log path), make 
sure you leave a white space in between quotes.}:
\begin{verbatim}
 executor = echo ' %s %s ' | mail -s 'log4tailer alert' -t youremail@hostname.com
\end{verbatim}
If sendmail sends email to your localhost, then you could read the email easily by using the 
famous command line client mutt for example. It's important to note, that you can daemonize 
\logftailer{} in that case as well:
\begin{cmd}
 log4tail --no-mail-silence --executable -c configfile.txt /var/log/out.log
\end{cmd}
That means that log4tailer will be a daemon monitoring the out.log and sending email by using 
the mail Linux command line. 

\item You can always make \logftailer{} to report you in a file every some minutes or 
activate the cornermark notifications (see section \ref{sec:cornermark}). 
Both features are really nice to activate them when 
you need to go for a break. You can setup the cornermark feature with the \emph{cornermark}
parameter specifying a time in seconds bit longer than the time you'll be out of your desktop
to avoid the mark going away. The marks stay in the terminal for the time you specify.
\end{itemize}


\newpage

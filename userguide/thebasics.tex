\pagestyle{fancy}
\part{The Basics}
\section{Introduction}
\logftailer{} aims to provide a different approach to traditional log tailers.
While the famous tail linux
command line tool does its job just well, log4tailer adds
some more advanced features, like throttling, user defined colors and email notification just 
to name a few. 
List of features:

\begin{itemize}
 \item Multitailing capability. It can tail multiple logs at a time
 \item Colors for every level: warn, info, debug, error and fatal
 \item Emphasize multiple targets (log traces) given regular expressions
 \item Follow log upon truncation by default
 \item User defined colors for each level
 \item Silent (daemonized) mode for full automatic log monitorization
 \item Throttling mode. Slow down the information being printed in the terminal 
 \item Inactivity log monitoring
 \item mail notification
 \item Every log with its own color scheme
 \item Freeze output upon specific and well defined conditions
 \item Cornermark notification, visual alert box in your terminal when you are out of your desktop
 \item Log reporting by email, to a file or standard output
 \item Tailing remote logs by means of ssh
 
\end{itemize}

\subsection{Why yet another tailer?}

Most people use tail -F to tail the logs these days. When debugging enterprise
class applications you cannot just follow (in many situations) what is going on
unless you go to the log, less it and check if something was wrong, or just
Ctrl-C tail program and scroll back. Human eye cannot distinguish or grab a line
out of thousands when that information is showed incredibly fast in the screen.
By providing colors, the human eye will discern and quickly identify specific
levels or lines. 

The \emph{tail} command line Unix/Linux tool has frustrated me many times, so I decided 
to write my own \emph{tailer} adding more functionalities as I needed them. There might be 
other tools out there, but to use them you must install them system wide or 
write your own regexes and remember the terminal escape characters to colorize the 
output. It wouldn't be the first time I see people writing perl one liners with complex regexes in order 
to tail logs. \logftailer{} is an 
standalone application that you can run from your /home directory if you desire with default common sense 
default colours.
Of course, you will need read access to those logs you want to monitor.  

\subsection{Why should I use it?}
Just to name a few advantages:

\begin{enumerate}
    \item \logftailer{} is opensource, free software.
    \item It runs in standalone mode.
At this very moment, no installation is necessary. 
If you are a sysadmin or an engineer monitoring logs in a network operation center, 
you can run \logftailer{} from its folder. Just untar and run the tailer. 

    \item Why do it in black and white, when you can do it with colors?

\end{enumerate}

\section{Installation and system requirements}

\subsection{System Requirements}

You'll need at least version 2.4 of Python installed in your system. 
Python is multiplatform and you can have a copy in \href{http://www.python.org}{http://www.python.org}.  

\logftailer{} has been tested in Linux servers and it should run in any xterm compatible terminal 
such as Putty, GNOME terminal and others.

\subsection{Installation}
\logftailer{} can run in standalone mode, so no installation is necessary if you don't want to. 
Untar, go to the \textbf{bin} folder and run ;-).
If for convenience, you want to install it, just type (having root access):
\begin{cmd}
 python setup.py install
\end{cmd}
and it will be installed system wide. For convenience, you can download the rpm, which has been packaged 
for SUSE Enterprise Server 10.

\subsection{Running the automated tests}
The tests can be run using \href{http://somethingaboutorange.com/mrl/projects/nose/0.11.3/}
{nosetests}. You'll need the \href{http://code.google.com/p/pymox/}{pymox}, 
\href{http://labix.org/mocker}{mocker} and 
\href{http://www.lag.net/paramiko/packages}{paramiko} packages to do that.
Alternatively, you can run:
\begin{cmd}
 make all
\end{cmd}
and it will install the required dependencies in the current directory plus a test script 
inside the bin folder. Then, just type:
\begin{cmd}
 make runtests
\end{cmd}

\section{Command line parameters}
The full command line list is as follows:
\begin{cmd}
 ./log4tail [-s silencemode] [-n numlines] [-t targets] [--throttle secs] [-i inactivityTime] 
                 [-m mailnotification] [--no-mail-silence] [-c configfile] fullpathToLogs
\end{cmd}


\newpage
